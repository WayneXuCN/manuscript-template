%% English Academic Paper Template
%% Compile: xelatex template-en.tex

\documentclass[macnew, en]{manuscript}

\setmainfont{Times New Roman}
\hypersetup{pdftitle={Paper Title}, pdfauthor={Author Name}}

\begin{document}

\title{Your Paper Title Here}
\author{
    Wenjie Xu\textsuperscript{a,b,*}, 
    Second Author\textsuperscript{a}, 
    Third Author\textsuperscript{b}\\[2pt]
    \small\textsuperscript{a}Institutes of Science and Development, CAS, Beijing 100190, China\\
    \small\textsuperscript{b}School of Public Policy and Management, UCAS, Beijing 100049, China\\
    \small\textsuperscript{*}Corresponding author
}
\date{}
\maketitle
\thispagestyle{plain}

\Abstract{Write your abstract here. The abstract should briefly summarize the main contributions, 
methodology, and results of your paper. Keep it concise and informative, typically within 150--250 words.}

\Keywords{Keyword 1; Keyword 2; Keyword 3; Keyword 4}

\section{Introduction}
\label{sec:intro}

Write your introduction here. Provide background context, state the research problem, 
and outline the paper structure \cite{example2023paper}.

The main contributions of this paper include:
\begin{itemize}
    \item Contribution 1
    \item Contribution 2
    \item Contribution 3
\end{itemize}

\section{Related Work}
\label{sec:related}

Review relevant literature and position your work.

\section{Methodology}
\label{sec:method}

\subsection{Problem Formulation}
\label{subsec:formulation}

Describe the problem formulation. Key equations should be numbered:

\begin{equation}
    f(x) = \sum_{i=1}^{n} \alpha_i \cdot g_i(x)
    \label{eq:example}
\end{equation}

\begin{definition}[Term Name]
    Provide the definition here.
\end{definition}

\begin{theorem}[Theorem Name]
    State the theorem here.
\end{theorem}

\begin{lemma}[Lemma Name]
    State the lemma here.
\end{lemma}

\subsection{Algorithm}
\label{subsec:algorithm}

\begin{algorithm}[t]
    \KwIn{Input parameters}
    \KwOut{Output result}
    Step 1 description\\
    \For{each iteration}{Update step}
    \Return{Final result}
    \caption{Algorithm Name}
    \label{alg:example}
\end{algorithm}

\Cref{alg:example} shows the main algorithm workflow.

\section{Experiments}
\label{sec:exp}

\subsection{Experimental Setup}
\label{subsec:setup}

Describe datasets, baseline methods, and evaluation metrics.

\begin{table}[t]
    \centering
    \caption{Dataset Statistics}
    \label{tab:data}
    \begin{tabular}{lcc}
        \toprule
        Metric & Dataset A & Dataset B \\
        \midrule
        Samples & XXX & XXX \\
        Features & XXX & XXX \\
        Classes & XXX & XXX \\
        \bottomrule
    \end{tabular}
\end{table}

\subsection{Results}
\label{subsec:results}

\begin{table}[t]
    \centering
    \caption{Performance Comparison}
    \label{tab:results}
    \begin{tabular}{lccc}
        \toprule
        Method & Metric 1 & Metric 2 & Metric 3 \\
        \midrule
        Baseline 1 & XX.X & XX.X & XX.X \\
        Baseline 2 & XX.X & XX.X & XX.X \\
        \textbf{Ours} & \textbf{XX.X} & \textbf{XX.X} & \textbf{XX.X} \\
        \bottomrule
    \end{tabular}
\end{table}

\Cref{tab:results} shows the comparison results. Our method outperforms all baselines.

\section{Conclusion}
\label{sec:conclusion}

Summarize the main findings and discuss future work directions.

\bibliography{references}

\end{document}