%% ============================================================================
%% English Academic Paper LaTeX Template
%% Compilation: xelatex template-en.tex
%% Dependencies: Requires XeLaTeX and the following fonts:
%%               - Times New Roman (Western font)
%%               - Source Han Serif/Heiti or system default Chinese fonts
%% ============================================================================

%% ----------------------------------------------------------------------------
%% Document Class Settings
%% ----------------------------------------------------------------------------
%% \documentclass[options]{manuscript}
%% Available Options:
%%   - Fontset: macnew (macOS default), windows, fandol (open source), ubuntu
%%   - Language: cn (Chinese), en (English)
%%   - Layout: twocolumn (two-column layout), review (review mode, double spacing)
%% Example: \documentclass[macnew, en, twocolumn]{manuscript}
%% ----------------------------------------------------------------------------
\documentclass[macnew, en]{manuscript}

%% ----------------------------------------------------------------------------
%% Western Font Configuration
%% ----------------------------------------------------------------------------
%% \setmainfont{font name} sets the main text font (serif)
%% Common Choices:
%%   - Times New Roman (academic standard)
%%   - TeX Gyre Termes (open source alternative to Times)
%%   - Latin Modern Roman
%% \setsansfont{font name} sets the sans-serif font (for headings)
%% \setmonofont{font name} sets the monospace font (for code)
%% ----------------------------------------------------------------------------
\setmainfont{Times New Roman}

%% ----------------------------------------------------------------------------
%% PDF Metadata Configuration
%% ----------------------------------------------------------------------------
%% Set PDF document properties for better indexing and management
%% Common Fields:
%%   - pdftitle: Paper title (displayed in PDF properties)
%%   - pdfauthor: Author name(s)
%%   - pdfsubject: Paper subject
%%   - pdfkeywords: Keywords
%% ----------------------------------------------------------------------------
\hypersetup{
    pdftitle={Paper Title},
    pdfauthor={Author Name}
}

%% ----------------------------------------------------------------------------
%% Document Body Begins
%% ----------------------------------------------------------------------------
\begin{document}

%% ----------------------------------------------------------------------------
%% Paper Title
%% ----------------------------------------------------------------------------
%% \title{title content}
%% Supports \thanks{} for footnotes, but the \authorinfo system below is recommended
%% ----------------------------------------------------------------------------
\title{Your Paper Title Here}

%% ----------------------------------------------------------------------------
%% Author Information Configuration (Modular System)
%% ----------------------------------------------------------------------------
%% Usage: \authorinfo[corresponding mark]{Name}{Affiliation marks}{Email}
%% Parameters:
%%   - [corresponding mark]: Optional, * indicates corresponding author (auto footnote)
%%   - Name: Author's name
%%   - Affiliation marks: Letter identifiers for affiliations, comma-separated for multiple, e.g., {a,b}
%%   - Email: Email address
%%
%% Examples:
%%   \authorinfo[*]{John Smith}{a}{john@example.edu}      % Corresponding author
%%   \authorinfo[]{Jane Doe}{a,b}{jane@example.edu}       % Non-corresponding, two affiliations
%%   \authorinfo[]{Bob Wilson}{b}{bob@example.edu}        % Non-corresponding, one affiliation
%% ----------------------------------------------------------------------------
\authorinfo[*]{Wenjie Xu}{a,b}{xuwenjie@example.edu.cn}
\authorinfo[]{Second Author}{a}{author2@example.edu.cn}
\authorinfo[]{Third Author}{b}{author3@example.edu.cn}

%% ----------------------------------------------------------------------------
%% Affiliation Configuration
%% ----------------------------------------------------------------------------
%% Usage: \affiliation{mark}{Institution name}
%% Parameters:
%%   - mark: Matches the affiliation marks in author configuration
%%   - Institution name: Full institution name and address
%%
%% Note: Mark order does not affect display order; alphabetical definition recommended
%% ----------------------------------------------------------------------------
\affiliation{a}{Institutes of Science and Development, CAS, Beijing 100190, China}
\affiliation{b}{School of Public Policy and Management, UCAS, Beijing 100049, China}

%% ----------------------------------------------------------------------------
%% Generate Title and Author Information
%% ----------------------------------------------------------------------------
%% \maketitle must be placed after all title and author information
%% \thispagestyle{plain} removes header on first page, keeps page number
%% ----------------------------------------------------------------------------
\maketitle
\thispagestyle{plain}

%% ----------------------------------------------------------------------------
%% Abstract and Keywords
%% ----------------------------------------------------------------------------
%% Use \Abstract{} command for the abstract
%% Abstract Requirements:
%%   - Briefly summarize research background, methods, results, and conclusions
%%   - Typically 150--250 words
%%   - No figures, tables, formulas, or citations
%%
%% Use \Keywords{} command for keywords
%% Format Requirements:
%%   - 3--5 keywords
%%   - Separated by semicolons (;)
%%   - English keywords should correspond to Chinese ones if applicable
%% ----------------------------------------------------------------------------
\Abstract{Write your abstract here. The abstract should briefly summarize the main contributions, 
methodology, and results of your paper. Keep it concise and informative, typically within 150--250 words.}

\Keywords{Keyword 1; Keyword 2; Keyword 3; Keyword 4}

%% ----------------------------------------------------------------------------
%% Main Body Begins
%% ----------------------------------------------------------------------------
%% Sections are automatically numbered; use \label{} to set cross-reference labels
%% Reference Methods:
%%   - \cref{label}: Smart reference (recommended, auto-adds "Section X/Fig.X/Table X" prefix)
%%   - \ref{label}: Number only
%%   - \pageref{label}: Page number
%% ----------------------------------------------------------------------------

\section{Introduction}
\label{sec:intro}

Write your introduction here. Provide background context, state the research problem, 
and outline the paper structure \cite{example2023paper}.

The main contributions of this paper include:
%% ----------------------------------------------------------------------------
%% Itemize Environment
%% ----------------------------------------------------------------------------
%% Available Options:
%%   - label: Custom item symbol, e.g., label=$\bullet$, label=\ding{108}
%%   - leftmargin: Adjust left margin
%%   - nosep: Compact mode (set by default)
%% ----------------------------------------------------------------------------
\begin{itemize}
    \item Contribution 1
    \item Contribution 2
    \item Contribution 3
\end{itemize}

\section{Related Work}
\label{sec:related}

Review relevant literature and position your work.

\section{Methodology}
\label{sec:method}

\subsection{Problem Formulation}
\label{subsec:formulation}

Describe the problem formulation. Key equations should be numbered:

%% ----------------------------------------------------------------------------
%% Math Environments
%% ----------------------------------------------------------------------------
%% equation environment: Numbered single-line equation
%% align environment: Multi-line aligned equations, use & to specify alignment position
%% gather environment: Multi-line centered equations
%% \[ ... \] or displaymath environment: Unnumbered equations
%% ----------------------------------------------------------------------------
\begin{equation}
    f(x) = \sum_{i=1}^{n} \alpha_i \cdot g_i(x)
    \label{eq:example}
\end{equation}

%% ----------------------------------------------------------------------------
%% Theorem-like Environments
%% ----------------------------------------------------------------------------
%% Available Environments:
%%   - theorem: Theorem
%%   - lemma: Lemma
%%   - proposition: Proposition
%%   - corollary: Corollary
%%   - definition: Definition
%%   - example: Example
%%   - assumption: Assumption
%%   - remark: Remark
%%
%% Usage: \begin{environment}[optional title] ... \end{environment}
%% ----------------------------------------------------------------------------
\begin{definition}[Term Name]
    Provide the definition here.
\end{definition}

\begin{theorem}[Theorem Name]
    State the theorem here.
\end{theorem}

\begin{lemma}[Lemma Name]
    State the lemma here.
\end{lemma}

\subsection{Algorithm}
\label{subsec:algorithm}

%% ----------------------------------------------------------------------------
%% Algorithm Pseudocode Environment
%% ----------------------------------------------------------------------------
%% From algorithm2e package
%% Common Commands:
%%   - \KwIn{input} / \KwOut{output}: Input/Output description
%%   - \For{condition}{operation}: for loop
%%   - \While{condition}{operation}: while loop
%%   - \If{condition}{operation}: conditional
%%   - \Return{value}: Return value
%%   - \\ or \; for line breaks
%%
%% Position Parameters: [t] = top, [b] = bottom, [h] = here
%% ----------------------------------------------------------------------------
\begin{algorithm}[t]
    \KwIn{Input parameters}
    \KwOut{Output result}
    Step 1 description\\
    \For{each iteration}{Update step}
    \Return{Final result}
    \caption{Algorithm Name}
    \label{alg:example}
\end{algorithm}

%% Use \cref for smart reference, automatically adds "Algorithm" prefix
\Cref{alg:example} shows the main algorithm workflow.

\section{Experiments}
\label{sec:exp}

\subsection{Experimental Setup}
\label{subsec:setup}

Describe datasets, baseline methods, and evaluation metrics.

%% ----------------------------------------------------------------------------
%% Table Environment
%% ----------------------------------------------------------------------------
%% Recommended Combination: tabular + booktabs (three-line table style)
%% Common Commands:
%%   - \toprule: Top thick line
%%   - \midrule: Middle thin line
%%   - \bottomrule: Bottom thick line
%%   - \cmidrule{l-r}: Partial horizontal line
%%   - l/c/r: Column alignment (left/center/right)
%%   - p{width}: Fixed width column with automatic line breaks
%%
%% Position Parameters: [t] = top, [b] = bottom, [h] = here, [!htbp] = force priority
%% ----------------------------------------------------------------------------
\begin{table}[t]
    \centering
    \caption{Dataset Statistics}
    \label{tab:data}
    \begin{tabular}{lcc}
        \toprule
        Metric & Dataset A & Dataset B \\
        \midrule
        Samples & XXX & XXX \\
        Features & XXX & XXX \\
        Classes & XXX & XXX \\
        \bottomrule
    \end{tabular}
\end{table}

\subsection{Results}
\label{subsec:results}

%% ----------------------------------------------------------------------------
%% Table Formatting
%% ----------------------------------------------------------------------------
%% Text bold: \textbf{text}
%% Math bold: \mathbf{symbol} or \bm{symbol} (requires bm package)
%% Best results are usually shown in bold
%% ----------------------------------------------------------------------------
\begin{table}[t]
    \centering
    \caption{Performance Comparison}
    \label{tab:results}
    \begin{tabular}{lccc}
        \toprule
        Method & Metric 1 & Metric 2 & Metric 3 \\
        \midrule
        Baseline 1 & XX.X & XX.X & XX.X \\
        Baseline 2 & XX.X & XX.X & XX.X \\
        \textbf{Ours} & \textbf{XX.X} & \textbf{XX.X} & \textbf{XX.X} \\
        \bottomrule
    \end{tabular}
\end{table}

%% Reference table
\Cref{tab:results} shows the comparison results. Our method outperforms all baselines.

\section{Conclusion}
\label{sec:conclusion}

Summarize the main findings and discuss future work directions.

%% ----------------------------------------------------------------------------
%% References
%% ----------------------------------------------------------------------------
%% \bibliographystyle{style} sets the bibliography style
%% Common styles (with natbib):
%%   - plainnat   : numbered by citation order, author-year format
%%   - abbrvnat   : abbreviated author names
%%   - unsrtnat   : unsorted, by appearance order
%%
%% \bibliography{filename} loads the .bib file
%% Do not include file extension
%% Citation Commands:
%%   - \cite{key}: Basic citation [1]
%%   - \citep{key}: Parenthetical citation (Author, Year) - requires natbib
%%   - \citet{key}: Textual citation Author (Year) - requires natbib
%% ----------------------------------------------------------------------------
\bibliographystyle{plainnat}
\bibliography{references}

\end{document}
