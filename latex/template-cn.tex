%% ============================================================================
%% 中文学术论文 LaTeX 模板
%% 编译方式: xelatex template-cn.tex
%% 依赖: 需要安装 XeLaTeX 和以下字体(macOS 默认已安装):
%%       - Times New Roman (西文字体)
%%       - 思源宋体/黑体 或系统默认中文字体
%% ============================================================================

%% ----------------------------------------------------------------------------
%% 文档类设置
%% ----------------------------------------------------------------------------
%% \documentclass[选项]{manuscript}
%% 可用选项:
%%   - 字体集: macnew (macOS默认), windows (Windows), fandol (开源), ubuntu
%%   - 语言: cn (中文), en (英文)
%%   - 版式: twocolumn (双栏排版), review (审稿模式,双倍行距)
%% 示例: \documentclass[macnew, cn, twocolumn]{manuscript}
%% ----------------------------------------------------------------------------
\documentclass[macnew, cn]{manuscript}

%% ----------------------------------------------------------------------------
%% 西文字体配置
%% ----------------------------------------------------------------------------
%% \setmainfont{字体名} 设置正文字体(衬线体)
%% 常用选择:
%%   - Times New Roman (学术论文标准)
%%   - TeX Gyre Termes (Times 的开源替代)
%%   - Latin Modern Roman
%% \setsansfont{字体名} 设置无衬线字体(用于标题等)
%% \setmonofont{字体名} 设置等宽字体(用于代码)
%% ----------------------------------------------------------------------------
\setmainfont{Times New Roman}

%% ----------------------------------------------------------------------------
%% PDF 元数据配置
%% ----------------------------------------------------------------------------
%% 设置 PDF 文档属性,便于检索和管理
%% 常用字段:
%%   - pdftitle: 论文标题(显示在 PDF 属性中)
%%   - pdfauthor: 作者姓名
%%   - pdfsubject: 论文主题
%%   - pdfkeywords: 关键词
%% ----------------------------------------------------------------------------
\hypersetup{
    pdftitle={论文标题},
    pdfauthor={作者姓名}
}

%% ----------------------------------------------------------------------------
%% 文档开始
%% ----------------------------------------------------------------------------
\begin{document}

%% ----------------------------------------------------------------------------
%% 论文标题
%% ----------------------------------------------------------------------------
%% \title{标题内容}
%% 支持使用 \thanks{} 添加脚注,但推荐使用下方的 \authorinfo 系统
%% ----------------------------------------------------------------------------
\title{论文标题}

%% ----------------------------------------------------------------------------
%% 作者信息配置(模块化系统)
%% ----------------------------------------------------------------------------
%% 用法: \authorinfo[通讯标记]{姓名}{单位标记}{邮箱}
%% 参数说明:
%%   - [通讯标记]: 可选,* 表示通讯作者,将自动添加脚注
%%   - 姓名: 作者姓名
%%   - 单位标记: 字母标识,用于关联单位,多个单位用逗号分隔如 {a,b}
%%   - 邮箱: 电子邮箱地址
%%
%% 示例:
%%   \authorinfo[*]{张三}{a}{zhangsan@example.edu.cn}  % 通讯作者
%%   \authorinfo[]{李四}{a,b}{lisi@example.edu.cn}     % 非通讯,两个单位
%%   \authorinfo[]{王五}{b}{wangwu@example.edu.cn}     % 非通讯,一个单位
%% ----------------------------------------------------------------------------
\authorinfo[*]{徐文杰}{a,b}{xuwenjie@example.edu.cn}
\authorinfo[]{第二作者}{a}{author2@example.edu.cn}
\authorinfo[]{第三作者}{b}{author3@example.edu.cn}

%% ----------------------------------------------------------------------------
%% 单位信息配置
%% ----------------------------------------------------------------------------
%% 用法: \affiliation{标记}{单位名称}
%% 参数说明:
%%   - 标记: 与作者配置中的单位标记对应
%%   - 单位名称: 完整的机构名称和地址
%%
%% 注意: 标记顺序不影响显示顺序,建议按字母顺序定义
%% ----------------------------------------------------------------------------
\affiliation{a}{中国科学院科技战略咨询研究院,北京 100190}
\affiliation{b}{中国科学院大学公共政策与管理学院,北京 100049}

%% ----------------------------------------------------------------------------
%% 生成标题和作者信息
%% ----------------------------------------------------------------------------
%% \maketitle 必须放在所有标题和作者信息之后
%% \thispagestyle{plain} 移除首页页眉,保留页码
%% ----------------------------------------------------------------------------
\maketitle
\thispagestyle{plain}

%% ----------------------------------------------------------------------------
%% 摘要和关键词
%% ----------------------------------------------------------------------------
%% 使用 \Abstract{} 命令撰写摘要
%% 摘要要求:
%%   - 简要概括研究背景、方法、结果和结论
%%   - 字数通常控制在 150--250 字之间
%%   - 不使用图表、公式和参考文献引用
%%
%% 使用 \Keywords{} 命令列出关键词
%% 格式要求:
%%   - 3--5 个关键词
%%   - 用分号 (;) 分隔
%%   - 中英文关键词可对应排列
%% ----------------------------------------------------------------------------
\Abstract{在此撰写摘要。摘要应简要概括论文的主要贡献、方法和结果,通常在 150--250 字之间。}

\Keywords{关键词 1; 关键词 2; 关键词 3; 关键词 4}

%% ----------------------------------------------------------------------------
%% 正文开始
%% ----------------------------------------------------------------------------
%% 章节自动编号,使用 \label{} 设置交叉引用标签
%% 引用方式:
%%   - \cref{标签} 智能引用(推荐,自动添加"第X章/图X/表X"等前缀)
%%   - \ref{标签} 仅显示编号
%%   - \pageref{标签} 显示页码
%% ----------------------------------------------------------------------------

\section{引言}
\label{sec:intro}

在此撰写引言部分。介绍研究背景、问题描述和论文结构 \cite{example2023paper}。

本模板支持中英文混排(Chinese and English mixed typesetting),数学公式和术语可以自然切换。

主要贡献包括:
%% ----------------------------------------------------------------------------
%% 无序列表环境
%% ----------------------------------------------------------------------------
%% 可用选项:
%%   - label: 自定义项目符号,如 label=$\bullet$, label=\ding{108}
%%   - leftmargin: 左边距调整
%%   - nosep: 紧凑模式(已默认设置)
%% ----------------------------------------------------------------------------
\begin{itemize}
    \item 贡献 1
    \item 贡献 2
    \item 贡献 3
\end{itemize}

\section{相关工作}
\label{sec:related}

综述相关文献,将你的工作定位在现有研究中。

\section{方法}
\label{sec:method}

\subsection{问题建模}
\label{subsec:formulation}

描述问题建模。关键公式应编号:

%% ----------------------------------------------------------------------------
%% 数学公式环境
%% ----------------------------------------------------------------------------
%% equation 环境: 带编号的单行公式
%% align 环境: 多行对齐公式,使用 & 指定对齐位置
%% gather 环境: 多行居中公式
%% \[ ... \] 或 displaymath 环境: 无编号公式
%% ----------------------------------------------------------------------------
\begin{equation}
    f(x) = \sum_{i=1}^{n} \alpha_i \cdot g_i(x)
    \label{eq:example}
\end{equation}

%% ----------------------------------------------------------------------------
%% 定理类环境
%% ----------------------------------------------------------------------------
%% 可用环境:
%%   - theorem: 定理
%%   - lemma: 引理
%%   - proposition: 命题
%%   - corollary: 推论
%%   - definition: 定义
%%   - example: 示例
%%   - assumption: 假设
%%   - remark: 注
%%
%% 用法: \begin{环境名}[可选标题] ... \end{环境名}
%% ----------------------------------------------------------------------------
\begin{definition}[术语名称]
    在此给出定义。
\end{definition}

\begin{theorem}[定理名称]
    在此陈述定理内容。
\end{theorem}

\begin{lemma}[引理名称]
    在此陈述引理内容。
\end{lemma}

\subsection{算法}
\label{subsec:algorithm}

%% ----------------------------------------------------------------------------
%% 算法伪代码环境
%% ----------------------------------------------------------------------------
%% 来自 algorithm2e 宏包
%% 常用命令:
%%   - \KwIn{输入} / \KwOut{输出}: 输入输出说明
%%   - \For{条件}{操作}: for 循环
%%   - \While{条件}{操作}: while 循环
%%   - \If{条件}{操作}: 条件判断
%%   - \Return{值}: 返回值
%%   - \\ 或 \; 换行
%%
%% 位置参数 [t]: 置顶 (top), [b]: 置底 (bottom), [h]: 此处 (here)
%% ----------------------------------------------------------------------------
\begin{algorithm}[t]
    \KwIn{输入参数}
    \KwOut{输出结果}
    步骤 1 描述\\
    \For{每次迭代}{更新步骤}
    \Return{最终结果}
    \caption{算法名称}
    \label{alg:example}
\end{algorithm}

%% 使用 \cref 进行智能引用,自动添加"算法"前缀
\cref{alg:example} 展示了主要算法流程。

\section{实验}
\label{sec:exp}

\subsection{实验设置}
\label{subsec:setup}

描述数据集、基线方法和评价指标。

%% ----------------------------------------------------------------------------
%% 表格环境
%% ----------------------------------------------------------------------------
%% 推荐组合: tabular + booktabs(三线表)
%% 常用命令:
%%   - \toprule: 顶部粗线
%%   - \midrule: 中间细线
%%   - \bottomrule: 底部粗线
%%   - \cmidrule{l-r}: 部分横线
%%   - l/c/r: 列对齐方式(左/中/右)
%%   - p{宽度}: 固定宽度列,自动换行
%%
%% 位置参数 [t]: 置顶, [b]: 置底, [h]: 此处, [!htbp]: 强制优先
%% ----------------------------------------------------------------------------
\begin{table}[t]
    \centering
    \caption{数据集统计}
    \label{tab:data}
    \begin{tabular}{lcc}
        \toprule
        指标 & 数据集 A & 数据集 B \\
        \midrule
        样本数 & XXX & XXX \\
        特征数 & XXX & XXX \\
        类别数 & XXX & XXX \\
        \bottomrule
    \end{tabular}
\end{table}

\subsection{实验结果}
\label{subsec:results}

%% ----------------------------------------------------------------------------
%% 表格内格式
%% ----------------------------------------------------------------------------
%% 文本加粗: \textbf{文本}
%% 数学加粗: \mathbf{符号} 或 \bm{符号}(需要 bm 宏包)
%% 最佳结果通常用加粗表示
%% ----------------------------------------------------------------------------
\begin{table}[t]
    \centering
    \caption{性能对比}
    \label{tab:results}
    \begin{tabular}{lccc}
        \toprule
        方法 & 指标 1 & 指标 2 & 指标 3 \\
        \midrule
        基线 1 & XX.X & XX.X & XX.X \\
        基线 2 & XX.X & XX.X & XX.X \\
        \textbf{本文方法} & \textbf{XX.X} & \textbf{XX.X} & \textbf{XX.X} \\
        \bottomrule
    \end{tabular}
\end{table}

%% 引用表格
\cref{tab:results} 展示了对比结果。结果表明本文方法优于所有基线方法。

\section{结论}
\label{sec:conclusion}

总结主要发现并讨论未来工作方向。

%% ----------------------------------------------------------------------------
%% 附录
%% ----------------------------------------------------------------------------
%% 使用 \appendix 命令开启附录部分
%% 此后章节自动编号为 A, B, C...,子章节为 A.1, A.2...
%%
%% 注意事项:
%%   - \appendix 命令应放在所有正文章节之后
%%   - 每个附录使用 \section 开始
%%   - 附录内部仍可使用 \subsection 和 \subsubsection
%%   - 附录标题前缀默认为"附录"(中文)或"Appendix"(英文)
%% ----------------------------------------------------------------------------
\appendix

\section{补充证明}
\label{app:proof}

本节提供文中定理的详细证明过程。

\subsection{定理 1 的详细证明}

\begin{theorem}[收敛性定理]
\label{thm:convergence}
    假设目标函数 $f(x)$ 满足 $L$-光滑条件,且学习率 $\eta \leq \frac{1}{L}$,则算法 1 保证在 $T$ 次迭代后收敛:
    \begin{equation}
        \min_{t \in [T]} \|\nabla f(x_t)\|^2 \leq \frac{2(f(x_0) - f^*)}{\eta T}
    \end{equation}
\end{theorem}

\begin{proof}
    根据 $L$-光滑性的定义,对于任意 $x, y$ 有:
    \begin{equation}
        f(y) \leq f(x) + \nabla f(x)^\top(y-x) + \frac{L}{2}\|y-x\|^2
    \end{equation}
    
    令 $y = x_{t+1} = x_t - \eta \nabla f(x_t)$,代入上式得:
    \begin{align}
        f(x_{t+1}) &\leq f(x_t) - \eta \|\nabla f(x_t)\|^2 + \frac{L\eta^2}{2}\|\nabla f(x_t)\|^2 \\
        &= f(x_t) - \eta\left(1 - \frac{L\eta}{2}\right)\|\nabla f(x_t)\|^2
    \end{align}
    
    当 $\eta \leq \frac{1}{L}$ 时,$1 - \frac{L\eta}{2} \geq \frac{1}{2}$,因此:
    \begin{equation}
        f(x_{t+1}) \leq f(x_t) - \frac{\eta}{2}\|\nabla f(x_t)\|^2
    \end{equation}
    
    对 $t = 0, 1, \ldots, T-1$ 求和:
    \begin{equation}
        f(x_T) \leq f(x_0) - \frac{\eta}{2}\sum_{t=0}^{T-1}\|\nabla f(x_t)\|^2
    \end{equation}
    
    整理并注意到 $f(x_T) \geq f^*$,即得证。
\end{proof}

\section{补充实验}
\label{app:experiment}

\subsection{在不同数据集上的性能对比}

\cref{tab:appendix-results} 展示了本文方法在多个公开数据集上的详细性能对比。

\begin{table}[h]
    \centering
    \caption{不同数据集上的性能对比(准确率 \%)}
    \label{tab:appendix-results}
    \begin{tabular}{lcccc}
        \toprule
        方法 & CIFAR-10 & CIFAR-100 & ImageNet & 平均 \\
        \midrule
        ResNet-50 & 94.2 & 74.5 & 76.1 & 81.6 \\
        ResNet-101 & 95.1 & 77.3 & 78.5 & 83.6 \\
        ViT-B/16 & 96.2 & 81.4 & 80.2 & 85.9 \\
        \textbf{本文方法} & \textbf{97.1} & \textbf{83.2} & \textbf{82.8} & \textbf{87.7} \\
        \bottomrule
    \end{tabular}
\end{table}

\subsection{超参数敏感性分析}

我们对关键超参数进行了敏感性分析。\cref{fig:hyperparam} 展示了不同学习率和批量大小对模型性能的影响。

%% 示例:插入图片的代码框架(需要准备图片文件)
%% \begin{figure}[h]
%%     \centering
%%     \includegraphics[width=0.8\textwidth]{figures/hyperparam_analysis.pdf}
%%     \caption{超参数敏感性分析}
%%     \label{fig:hyperparam}
%% \end{figure}

实验结果表明,当学习率在 $[10^{-4}, 10^{-3}]$ 范围内,批量大小在 $[32, 128]$ 范围内时,模型性能相对稳定。

%% ----------------------------------------------------------------------------
%% 参考文献
%% ----------------------------------------------------------------------------
%% \bibliographystyle{样式名} 设置参考文献样式
%% 常用样式(配合 natbib 使用):
%%   - plainnat   : 按引用顺序编号,作者-年份格式
%%   - abbrvnat   : 缩写作者名
%%   - unsrtnat   : 不排序,按引用出现顺序
%%
%% \bibliography{文件名} 加载 .bib 文件
%% 不需要写扩展名
%% 引用命令:
%%   - \cite{key}: 基础引用 [1]
%%   - \citep{key}: 括号引用 (Author, Year) - 需要 natbib
%%   - \citet{key}: 文本引用 Author (Year) - 需要 natbib
%% ----------------------------------------------------------------------------
\bibliographystyle{plainnat}
\bibliography{references}

\end{document}
