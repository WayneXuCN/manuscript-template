%% 中文学术论文模板
%% 编译: xelatex template-cn.tex

\documentclass[macnew, cn]{manuscript}

%% 西文字体
\setmainfont{Times New Roman}

%% PDF 元数据
\hypersetup{pdftitle={论文标题}, pdfauthor={作者姓名}}

\begin{document}

\title{论文标题}
\author{
    徐文杰\textsuperscript{a,b,*}, 
    第二作者\textsuperscript{a}, 
    第三作者\textsuperscript{b}\\[2pt]
    \small\textsuperscript{a}中国科学院科技战略咨询研究院,北京 100190\\
    \small\textsuperscript{b}中国科学院大学公共政策与管理学院,北京 100049\\
    \small\textsuperscript{*}通讯作者
}
\date{}
\maketitle
\thispagestyle{plain}

\begin{abstract}
    在此撰写摘要。摘要应简要概括论文的主要贡献、方法和结果,通常在 150--250 字之间。
    \keywords{关键词 1; 关键词 2; 关键词 3; 关键词 4}
\end{abstract}

\section{引言}
\label{sec:intro}

在此撰写引言部分。介绍研究背景、问题描述和论文结构 \cite{example2023paper}。

本模板支持中英文混排(Chinese and English mixed typesetting),数学公式和术语可以自然切换。

主要贡献包括:
\begin{itemize}
    \item 贡献 1
    \item 贡献 2
    \item 贡献 3
\end{itemize}

\section{相关工作}
\label{sec:related}

综述相关文献,将你的工作定位在现有研究中。

\section{方法}
\label{sec:method}

\subsection{问题建模}
\label{subsec:formulation}

描述问题建模。关键公式应编号:

\begin{equation}
    f(x) = \sum_{i=1}^{n} \alpha_i \cdot g_i(x)
    \label{eq:example}
\end{equation}

\begin{definition}[术语名称]
    在此给出定义。
\end{definition}

\begin{theorem}[定理名称]
    在此陈述定理内容。
\end{theorem}

\begin{lemma}[引理名称]
    在此陈述引理内容。
\end{lemma}

\subsection{算法}
\label{subsec:algorithm}

\begin{algorithm}[t]
    \KwIn{输入参数}
    \KwOut{输出结果}
    步骤 1 描述\\
    \For{每次迭代}{更新步骤}
    \Return{最终结果}
    \caption{算法名称}
    \label{alg:example}
\end{algorithm}

\cref{alg:example} 展示了主要算法流程。

\section{实验}
\label{sec:exp}

\subsection{实验设置}
\label{subsec:setup}

描述数据集、基线方法和评价指标。

\begin{table}[t]
    \centering
    \caption{数据集统计}
    \label{tab:data}
    \begin{tabular}{lcc}
        \toprule
        指标 & 数据集 A & 数据集 B \\
        \midrule
        样本数 & XXX & XXX \\
        特征数 & XXX & XXX \\
        类别数 & XXX & XXX \\
        \bottomrule
    \end{tabular}
\end{table}

\subsection{实验结果}
\label{subsec:results}

\begin{table}[t]
    \centering
    \caption{性能对比}
    \label{tab:results}
    \begin{tabular}{lccc}
        \toprule
        方法 & 指标 1 & 指标 2 & 指标 3 \\
        \midrule
        基线 1 & XX.X & XX.X & XX.X \\
        基线 2 & XX.X & XX.X & XX.X \\
        \textbf{本文方法} & \textbf{XX.X} & \textbf{XX.X} & \textbf{XX.X} \\
        \bottomrule
    \end{tabular}
\end{table}

\cref{tab:results} 展示了对比结果。结果表明本文方法优于所有基线方法。

\section{结论}
\label{sec:conclusion}

总结主要发现并讨论未来工作方向。

\bibliography{references}

\end{document}